\documentclass[a4paper]{article}
\usepackage[utf8]{inputenc}
\usepackage[margin=.9cm,tmargin=1.2cm]{geometry}
\usepackage[none]{hyphenat}
\usepackage{lipsum,xcolor,relsize,xspace,hyperref}

\setlength{\parindent}{0pt}

%\colorlet{hl}{black!70}
\definecolor{hl}{HTML}{800020}

\newcommand\hone{\fontfamily{phv}\fontseries{sluc}\fontsize{14}{16}\selectfont }
\newcommand\institute{\fontfamily{phv}\fontseries{bx}\fontsize{11}{14}\selectfont }
\newcommand\qualification{\fontfamily{phv}\fontseries{m}\fontshape{sl}\fontsize{10}{11}\selectfont }
\newcommand\stdate{\fontfamily{phv}\fontseries{m}\fontshape{sc}\fontsize{10}{11}\selectfont }
\newcommand\grade{\fontfamily{phv}\fontseries{sluc}\fontsize{10}{11}\selectfont\color{hl} }
\newcommand\info{\fontfamily{phv}\fontseries{sluc}\fontsize{10}{11}\selectfont\color{hl} }
\newcommand\boldinfo{\fontfamily{phv}\fontseries{b}\fontsize{10}{11}\selectfont\color{hl} }
\newcommand\itinfo{\fontfamily{phv}\itshape\fontsize{10}{11}\selectfont\color{hl} }

\newcommand\dash{ {\large\bf --} }
\newcommand\dateword{\hspace{1px}\fontfamily{phv}\fontseries{m}\fontshape{n}\fontsize{9}{10}\selectfont}
\newcommand\cpp{{C\nolinebreak[4]\hspace{-.15em}\raisebox{.4ex}{\relsize{-3}\bf +\hspace{-.15em}+}}\xspace}

\newcommand\skill\institute
\newcommand\planguage\desc%\stdate
\newcommand\boldplanguage{\fontfamily{phv}\fontseries{b}\fontsize{10}{11}\selectfont }
\newcommand\desc{\fontfamily{phv}\fontseries{sluc}\fontsize{10}{11}\selectfont }
\newcommand\module\planguage%{\fontfamily{phv}\fontseries{m}\fontshape{sl}\fontsize{10}{11}\selectfont }
\newcommand\referee{\fontfamily{phv}\fontseries{bx}\fontsize{10}{11}\selectfont }
\newcommand\sep{\,{\color{black}\bf\textbar}\,}
\newcommand\referenceonrequest{%
  \hrule\vspace{.7em}
  {\qualification References available on request.}
}
\newcommand\hl[1]{{\color{hl}#1}}

\begin{document}
\thispagestyle{empty}

\begin{minipage}{\textwidth}

   %TOP PANEL (TITLE)
  \begin{minipage}[t]{0.7\textwidth}
    {\fontfamily{phv}\fontseries{ubuc}\fontsize{22}{25}\selectfont%
      HARRY COOKE 
      \fontfamily{phv}\fontseries{sluc}\fontsize{18}{20}\selectfont\color{hl}
      \hspace{1pt}PhD*
    }\\[.2em]
    {\fontfamily{phv}\fontseries{sluc}\fontsize{12}{14}\selectfont\color{hl}%
      Software development\sep Data analysis
    }
  \end{minipage}\vspace{-.2em}%
  \begin{minipage}[t]{0.3\textwidth}
    \vspace{-1.65em}
    \raggedleft\info
      \vspace{-.3em}
      hcooke006@aol.com \\[0.1em]
      +44 7528 694569 \\[0.1em]
      \href{https://github.com/Hazza4569}{github.com/Hazza4569}
  \end{minipage}

\end{minipage}

\vspace{0.5em}
%\hrule
\vspace{0.5em}

\begin{minipage}[t]{0.64\textwidth}
  % LEFT COLUMN 
  \newcommand\intersectionspacing{1.3em}
  \newcommand\intersegmentspacing{-.3em}
  % ===== EDUCATION =====
  {\hone Education} \vspace{0.5em} \hrule \vspace{0.5em} 

  {\institute University of Birmingham}\\
  {\stdate 2019 - 2023} \dash 
  { \qualification PhD Particle Physics}\\
  {\info%
    Searching for rare Standard Model interactions with the
    ATLAS collaboration
  }\\
  {\info%
    Statistical analysis, machine learning, programming in \cpp and Python
  }\\
  {\itinfo%
    *completed PhD programme, graduating summer 2024
  }\\[\intersegmentspacing]

  {\institute University of Birmingham}\\ 
  {\stdate 2015 - 2019} \dash
  { \qualification%
    MSci Physics with Particle Physics and Cosmology \\
    Class I, Cum Laude
  }\\
  {\info Graduated first in year for experimental physics}\\
  {\info Awarded Bloodworth Prize (Y3) and Moreton Prize (Y4) for academic excellence}
  {\info and Tessella Prize for most innovative use of software in a Y4 project}
  \\[\intersegmentspacing]

  {\institute Hereford Sixth Form College}\\
  {\stdate 2013 - 2015} \dash
  { \qualification A-Levels:
  %{\grade
  A*s in physics, maths, further maths, computing
  }

  \vspace{\intersectionspacing}

  % ==== EXPERIENCE ======
  {\hone Experience} \vspace{0.5em} \hrule \vspace{0.5em} 

  {\institute University of Birmingham} \dash
  { \qualification Postgraduate Teaching Assistant}\\
  {\stdate {\dateword OCTOBER} 2019 - \dateword{MARCH} 2023}\\
  {\info Teaching in undergraduate computing and physics
  labs}\\[\intersegmentspacing]

  {\institute CERN} \dash
  { \qualification Summer Studentship}\\
  {\stdate {\dateword JUNE - AUGUST} 2018}\\
  {\info Developing a monitoring system for AMC13 modules in the CMS detector}\\
  {\info \url{http://cdlib.cern.ch/record/2640969}}\\[\intersegmentspacing]

  {\institute University of Birmingham} \dash
  { \qualification Maple TA Internship}\\
  {\stdate {\dateword JULY - AUGUST} 2017, {\dateword DECEMBER} 2017}\\
  {\info Authoring and testing maths questions for university science faculties}

  \vspace{\intersectionspacing}

  % ==== TECHNICAL SKILLS ======
  {\hone Technical Skills} \vspace{0.5em} \hrule \vspace{0.5em} 

  % > PROGRAMMING
  {\skill Software development}\\
  % > Primary languages
  {\info 10 years of experience and competent in multiple programming languages:}\\
  {\boldplanguage \cpp}\dash
  {\desc%
    Daily use throughout PhD for data analysis and algorithm design, working
    with large software frameworks. 8 years of experience.
  }\\
  {\boldplanguage Python}\dash
  {\desc%
    Regular use for scripting and data visualisation, used for larger personal
    projects, taught to Y3 undergraduate students. 7 years of experience.
  }\\
  {\boldplanguage JavaScript {\dateword (HTML, CSS)}}\dash
  {\desc%
    Built monitoring and visualisation tools for use with detector hardware.
    Some small personal web projects. 6 years of experience.
  }\\
  {\boldplanguage Visual Basic}\dash
  {\desc%
    Used during A-level computing, developed a simulation of Conway's Game of
    Life. 2 years experience.
  }\\
  % > Additional languages
  {\info Some experience with many more languages, including}\\
  {\planguage%
    Bash, Ruby, Java, Go, Rust, Kotlin, SQL, Matlab, Maple, Julia
  }\\[\intersegmentspacing]

  % > DATA ANALYSIS
  {\skill Data analysis}\\
  {\info Experienced with a number of analysis techniques:}\\
  {\desc%
    Analysis performed during the PhD required use and understanding of
    \qualification
    machine learning techniques, likelihood model building, maximum likelihood
    estimation, treatment of systematic and statistical uncertainties,
    hypothesis testing,
    \desc
    and more.
    Received formal training in statistical methods during undergraduate degree
    and in a postgraduate course.
  }\\[\intersegmentspacing]

  % > Misc
  {\skill Miscellaneous}\\
  {\desc Experienced with \hl{version-control software}, see a selection of projects on my
    \href{https://github.com/Hazza4569}{\color{hl}\underline{GitHub page}}.
    Adept in operation of \hl{unix-based systems}, for both work and personal use.
    Proficient in writing and typesetting of documents, particularly with
    {\LaTeX} -- created the official University of Birmingham
    \href{https://www.overleaf.com/edu/bham#templates}
    {\color{hl}\underline{overleaf templates}}.
  }\\

  \vspace{-.2em}
  \referenceonrequest

\end{minipage}%
\hfill
%\vrule
%\hfill
\begin{minipage}[t]{0.339\textwidth}
  \newcommand\interskillspacing{.3em}

  %RIGHT COLUMN
  % ===== Transferable skills =====
  {\hone Transferable Skills} \vspace{0.5em} \hrule \vspace{0.5em} 
  {\skill Project management}\\
  {\desc
    Independently planned and performed a 2.5-year-long physics analysis.
    Involved long-term planning and prioritisation of tasks to ensure that the analysis
    was completed within the timeline of the PhD.
  }\\[\interskillspacing]
  {\skill Teamwork and communication}\\
  {\desc
    Collaborated on many projects with colleagues at the University of
    Birmingham and internationally. Worked on an analysis with a team of
    $\sim$10 physicists to attain a world-leading result, meeting weekly to
    share updates with the team and distribute tasks. During a 1-year placement
    at CERN during the PhD, joined the L1Calo operations team and took week-long
    24/7 on-call shifts to solve problems during data-taking runs.
  }\\[\interskillspacing]
  {\skill Time management}\\
  {\desc
    Worked on many simultaneous projects during the PhD and learned to
    effectively split time between tasks, whilst also scheduling individual work
    time around meetings and other commitments.
  }\\[\interskillspacing]
  {\skill Leadership}\\
  {\desc
    During undergraduate studies, led a group of 14 students for a 3-month
    project, producing a 90-page document on the design of high-energy particle detectors.
    Leadership responsibilities included dividing and delegating tasks amongst
    small teams, setting deadlines, and organising and chairing regular
    meetings.
    Also took on leading roles in organising events such as a group Christmas
    dinner in Birmingham and a BBQ to welcome new arrivals to placement at CERN.
  }\\[\interskillspacing]
  {\skill Problem solving} \\
  {\desc
    As a physicist, problem solving is an essential skill, used in deriving
    solutions to mathematical problems, in implementing code to analyse very large
    datasets or simulate systems, and in optimising an analysis to best extract
    a result from data.
  }\\
  \vspace{0.5em}

  % ==== PERSONAL INTERESTS ======
  {\hone Personal Interests} \vspace{0.5em} \hrule \vspace{0.5em} 
  {\desc%
    Founded the University of Birmingham Benchball Society and served as
    president and then treasurer over the first 2 years. 
    Played tennis from the age of 11 and earned a Level-2 coaching qualification;
    spent 3 years coaching weekly sessions at local club. Enjoy
    working with computers, having built a custom PC for work, gaming, and more.
  }

\end{minipage}

%\vspace{1.2em}
%\hrule

\end{document}
